
\documentclass[letterpaper,11pt]{article} % Default font size and paper size

\usepackage{xunicode,xltxtra,url,parskip} % Formatting packages
\usepackage{fontspec}

\usepackage[usenames,dvipsnames]{xcolor} % Required for specifying custom colors

%\usepackage[big]{layaureo} % Margin formatting of the A4 page, an alternative to layaureo can be \usepackage{fullpage}
% To reduce the height of the top margin uncomment: \addtolength{\voffset}{-1.3cm}
%\usepackage[cm]{fullpage}
\usepackage[top=0.5in, bottom=0.5in, left=0.5in, right=0.5in]{geometry}

\usepackage{hyperref} % Required for adding links	and customizing them
\definecolor{linkcolour}{rgb}{0,0.2,0.6} % Link color
\hypersetup{colorlinks,breaklinks,urlcolor=linkcolour,linkcolor=linkcolour} % Set link colors throughout the document

\usepackage{tabulary}

\usepackage{array}
\newcolumntype{a}[1]{>{\raggedright\let\newline\\\arraybackslash\hspace{0pt}}m{#1}}
\newcolumntype{b}[1]{>{\centering\let\newline\\\arraybackslash\hspace{0pt}}m{#1}}
\newcolumntype{c}[1]{>{\raggedleft\let\newline\\\arraybackslash\hspace{0pt}}m{#1}}

\usepackage{titlesec} % Used to customize the \section command
\titleformat{\section}{\Large\scshape\raggedright}{}{0em}{}[\titlerule] % Text formatting of sections
\titlespacing{\section}{0pt}{3pt}{3pt} % Spacing around sections

\usepackage{fancyhdr}

\begin{document}

%\pagestyle{empty} % Removes page numbering
\pagestyle{fancy}
\pagenumbering{gobble}
\fancyhead[R]{\textbf{T. Mastrianni Perry}}
\thispagestyle{empty}

\font\fb=''[cmr10]'' % Change the font of the \LaTeX command under the skills section

%----------------------------------------------------------------------------------------
%	NAME AND CONTACT INFORMATION
%----------------------------------------------------------------------------------------

\par{\centering{\Huge Thomas Mastrianni \textsc{Perry}, Ph.D.}\par} % Your name
%\par{\centering{\Huge Thomas Mastrianni \textsc{Perry}, Ph.D.}\bigskip\par} % Your name
\par{\centering {\Large  \href{mailto:tomperry7@gmail.com}{tomperry7@gmail.com} | (518) 859 2623 | Santa Fe, NM | \textsc{github:} \href{https://www.github.com/tmrhombus}{tmrhombus}\par} }

\vspace{15pt}

%\par{\centering{Supplemental Information}\bigskip\par} % Your name

% \section{Personal Data}
% 
% \begin{tabular}{rl}
% \textsc{Address:} & 130 County Road 74, Tesuque, NM 87501 \\
% %\textsc{Address:} & 645 East Barcelona Rd., Santa Fe, NM 87505 \\
% %\textsc{Address:} & 119 Park Place, Schenectady, NY, 12305 \\
% %\textsc{Address:} & 26 rue de la Coulouvrenière, 1204, Geneva, Switzerland \\
% \textsc{Phone:} & 518 859 2623 \\
% \textsc{email:} & \href{mailto:tomperry7@gmail.com}{tomperry7@gmail.com} \\
% %\textsc{email:} & \href{mailto:tperry@sfprep.org}{tperry@sfprep.org} \\
% %\textsc{Skype:} & tmrhombus \\
% \end{tabular}
% 
% %----------------------------------------------------------------------------------------
% %	EDUCATION
% %----------------------------------------------------------------------------------------
% 
% \section{Education}
% 
% \begin{tabular}{rl}	
% 2016 & {\bf Ph.D. in Physics} (particle physics minor) University of Wisconsin--Madison, Madison, WI, USA \\
%  {}  & \emph{A measurement of $Wb\overline{b}$ production and a search for monophoton signals of dark matter}\\
%  {}  & \emph{ using the CMS detector at the CERN LHC} \\
% 
% 2009 & {\bf B.S. in Physics} (astrophysics minor) Union College, Schenectady, NY, USA \\
%  {}  & Magna Cum Laude with departmental honors \\
%  {}  & \emph{Multi-frequency VLBI imaging of two compact symmetric objects} \\
% \end{tabular}


%----------------------------------------------------------------------------------------
%	Publications 
%----------------------------------------------------------------------------------------

\section{Selected Publications [\href{https://inspirehep.net/author/profile/T.Perry.1}{712 total, h index = 105}]}

\begin{itemize}

 % Monophoton 12.9/fb paper
\item CMS Collaboration, 
  ``Search for new physics in the monophoton final state in proton-proton collisions at $\sqrt{s}=$ 13 TeV'',
  \emph{JHEP} (2017) 10
  \texttt{doi: 10.1007/JHEP10(2017)073},
  \texttt{\href{http://arxiv.org/abs/1706.03794}{arXiv:1706.03794}}.

 % 8 TeV Wbb paper
\item CMS Collaboration, 
  ``Measurement of the production cross section of a W boson
  in association with two b jets in pp collisions at $\sqrt{s}=$ 8 TeV'',
  \emph{Eur. Phys. J. C} {\bf 77} (2017) 2, 
  \texttt{doi: 10.1140/epjc/s10052-016-4573-z},
  \texttt{\href{http://arxiv.org/abs/1608.07561}{arXiv:1608.07561}}.

 % 2016 Monophoton EXO PAS 12.9/fb
\item CMS Collaboration, 
  ``Search for dark matter and graviton produced in
  association with a photon in pp collisions at $\sqrt{s}=$ 13 TeV
  with an integrated luminosity of 12.9fb$^{-1}$'',
  2016,
  \texttt{CMS-PAS-EXO-16-039}, \\
  \texttt{\href{http://cds.cern.ch/record/2205148}{http://cds.cern.ch/record/2205148}}.

 % 2016 EXO Monophoton PAS 2.3/fb
\item CMS Collaboration, 
  ``Search for Dark Matter and Large Extra Dimensions in the
  gamma + MET final state in pp Collisions at $\sqrt{s}=$ 13 TeV'',
  2016,
  \texttt{CMS-PAS-EXO-16-014},
  \texttt{\href{http://cds.cern.ch/record/2160229}{http://cds.cern.ch/record/2160229}}.

 % 2016 SMP Z(nunu)gamma PAS 2.3/fb
\item CMS Collaboration, 
  ``Measurement of the production cross section for pp to 
  Z(nu nu) gamma at $\sqrt{s}=$ 13 TeV at CMS'', 2016,
  \texttt{CMS-PAS-SMP-16-004},
  \texttt{\href{http://cds.cern.ch/record/2204922}{http://cds.cern.ch/record/2204922}}.

 % 7 TeV Wbb paper
\item CMS Collaboration, 
  ``Measurement of the production cross section for a W boson
  and two b jets in pp collisions at $\sqrt{s}=$ 7 TeV'',
  \emph{Phys. Lett. B} {\bf 735} (2014) 204, 
  \texttt{doi: 10.1016/j.physletb.2014.06.041},
  \texttt{\href{http://arxiv.org/abs/1312.6608}{arXiv:1312.6608}}.

 % Marr VLBA paper
\item J.M. Marr, T.M. Perry, J. Read, G.B. Taylor, A.O. Morris, 
  ``Multi-frequency Optical-depth Maps and the Case for
  Free-Free Absorption in Two Compact Symmetric Radio
  Sources: The CSO Candidate J1324 + 4048 and the CSO J0029
  + 3457'',
  \emph{Astrophys. J.} {\bf 780} (2014) 178, 
  \texttt{doi: 10.1088/0004-637X/780/2/178},
  \texttt{\href{http://arxiv.org/abs/1311.5762}{arXiv:1311.5762}}.

 % Trigger TDR paper
\item CMS Collaboration, 
  ``CMS Technical Design Report for the Level-1 Trigger Upgrade'',
  2013,  \\
  \texttt{CERN-LHCC-2013-011, CMS-TDR-012},
  \texttt{\href{https://cds.cern.ch/record/1556311}{https://cds.cern.ch/record/1556311}}.

\end{itemize}


%----------------------------------------------------------------------------------------
%	Fellowships/Prizes/Awards
%----------------------------------------------------------------------------------------

\section{Fellowships/Prizes/Awards}

\begin{tabular}{rl}	

2017 & Finalist - Famelab Switzerland \\
 {}  & \emph{The Swiss national finals for an international science communication competition} \\

2015 & CMS Create First Prize - CMS Collaboration \\
 {}  & \emph{A competition to create a new public exhibit located at CMS} \\

2009 & Minerva Fellowship - Thomas McEvoy and Harold Fried \\
 {}  & \emph{An eleven-month, \$10,000 fellowship to live and volunteer in rural Uganda} \\

2009 & Undergraduate Student Prize - Astronomical Society of New York \\
 {}  & \emph{Annual award for the best astronomy-related undergraduate research paper} \\

2009 & Josephine Daggett Award - Presidential Award - Union College \\
 {}  & \emph{In recognition of the student voted by the faculty to be ``of the best character and conduct''} \\

2008 & Booth-Ferris Fellowship - Edward Jones \\
 {}  & \emph{A private \$3000 fellowship to perform undergraduate research} \\

2007 & James Henry Turnbill Award - Union College \\
 {}  & \emph{In recognition of the ``best sophomore physics student''} 

\end{tabular}

{\bf Honor Societies} 2009 Phi Beta Kappa - National Honor Society ; 2009 Sigma Xi - National Research Honor Society ; 2007 Sigma Pi Sigma - National Physics Honor Society ; 2006, 2007, 2008, 2009 Union College Dean's List


%%----------------------------------------------------------------------------------------
%%	Honor Societies
%%----------------------------------------------------------------------------------------
%
%\section{Honor Societies}
%
%\begin{tabular}{rl}	
%2009 & Phi Beta Kappa - National Honor Society \\
%2009 & Sigma Xi - National Research Honor Society  \\
%2007 & Sigma Pi Sigma - National Physics Honor Society \\
%2006, 2007, 2008, 2009 & Union College Dean's List
%\end{tabular}


%----------------------------------------------------------------------------------------
%	Work Experience
%----------------------------------------------------------------------------------------
\section{Work Experience}


\begingroup
\fontsize{10pt}{12pt}\selectfont

\begin{tabulary}{\textwidth}{RL}	

Jun 2020 - & Math Teacher (Algebra I)/Tutor, Supervisor: Aiyana Pendleton, 505-982-1829, \href{mailto:apendleton@sfprep.org}{apendleton@sfprep.org}  \\
 present    &  Santa Fe Preparatory School, 1101 Camino Cruz Blanca, Santa Fe, NM, 87505 \\ 
%
\vspace{5pt} \\

Aug 2019 -     & Math Teacher (Calculus/Algebra II), Supervisor: James Leonard, 505-795-7514, \href{mailto:jleonard@sfprep.org}{jleonard@sfprep.org} \\
 May 2020       & Santa Fe Preparatory School, 1101 Camino Cruz Blanca, Santa Fe, NM, 87505 \\ %\\

\vspace{5pt} \\

Jan 2019 -     & Carpenter, Supervisor: Bob Fuller, 781-248-6446, \href{mailto:info@southshoreboatworks.com}{info@southshoreboatworks.com} \\
 May 2019      & South Shore Boatworks, 22 Industrial Blvd., Hanson, MA, 02341 \\ %\\

\vspace{5pt} \\

%May 2017 -     & Operations Manager for the CMS Hadronic Calorimeter, Supervisor: Paolo Rumerio, 205-348-2565, \href{mailto:rumerio@cern.ch}{rumerio@cern.ch} \\
%%May 2017 -     & Operations Manager for the CMS Hadronic Calorimeter, Supervisor: Paolo Rumerio, \href{mailto:rumerio@cern.ch}{rumerio@cern.ch} \\
%% Aug 2018      & CERN, Route de Meyrin 385, Meyrin, CH, 1204 \\ %\\ 
May 2017 -     & Operations Manager for the CMS Hadronic Calorimeter, Supervisor: Paolo Rumerio, 205-348-2565, \\
 Aug 2018      & \href{mailto:rumerio@cern.ch}{rumerio@cern.ch} CERN, Route de Meyrin 385, Meyrin, CH, 1204 \\ %\\ 

\vspace{5pt} \\

Nov 2016 -     & Postdoctoral Fellow, Supervisor: Ted Kolberg, 850-445-6866, \href{mailto:tkolberg@hep.fsu.edu}{tkolberg@hep.fsu.edu} \\
 Oct 2018      & Florida State University, Department of Physics, 77 Chieftain Way, Tallahassee, FL, 32306 \\ %\\ 

\vspace{5pt} \\

Sept 2016 -    & Postdoctoral Fellow, Supervisor: Wesley Smith, 608-251-9610, \href{mailto:wsmith@hep.wisc.edu}{wsmith@hep.wisc.edu} \\
 Oct 2016      & CERN, Route de Meyrin 385, Meyrin, CH, 1204 \\ %\\

\vspace{5pt} \\

Jun 2013 -     & Research Assistant, Supervisor: Wesley Smith, 608-251-9610, \href{mailto:wsmith@hep.wisc.edu}{wsmith@hep.wisc.edu} \\
 Aug 2016      & CERN, Route de Meyrin 385, Meyrin, CH, 1204 \\ %\\

\vspace{5pt} \\

Sep 2013 -     & Teaching Assistant (Physics 103/104), Supervisor: Wesley Smith, 608-251-9610, \href{mailto:wsmith@hep.wisc.edu}{wsmith@hep.wisc.edu} \\
 May 2014      & University of Wisconsin--Madison, Department of Physics, 1150 University Ave., Madison, WI, 53706 \\ %\\

\vspace{5pt} \\

Jun 2012 -     & Research Assistant, Supervisor: Wesley Smith, 608-251-9610, \href{mailto:wsmith@hep.wisc.edu}{wsmith@hep.wisc.edu} \\
 Aug 2013      & CERN, Route de Meyrin 385, Meyrin, CH, 1204 \\ %\\

\vspace{5pt} \\

Sep 2011 -     & Teaching Assistant (Physics 103/104), Supervisor: Wesley Smith, 608-251-9610, \href{mailto:wsmith@hep.wisc.edu}{wsmith@hep.wisc.edu} \\
 May 2012      & University of Wisconsin--Madison, Department of Physics, 1150 University Ave., Madison, WI, 53706 \\ %\\

\vspace{5pt} \\

Jun 2011 -     & Research Assistant, Supervisor: Albrecht Karle, 608-262-3945, \href{mailto:albrecht.karle@icecube.wisc.edu}{albrecht.karle@icecube.wisc.edu} \\
 Aug 2011      & IceCube Collaboration, 222 West Washington Ave \#500, Madison, WI, 53703 \\ %\\

\vspace{5pt} \\

Nov 2010 -     & Chemical Engineering Technician, Supervisor: Sarah Genovese, \href{mailto:genovese@ge.com}{genovese@ge.com} \\
 May 2011      & General Electric Global Research Center, 1 Research Circle, Niskayuna, NY, 12309 \\ %\\

\vspace{5pt} \\

Jul 2009 -     & A-level Physics Teacher \\
 Apr 2010      & St. Bernard's Secondary School, Kiswera, Uganda (Masaka District) \\
 {}            & \emph{As part of the Minerva Fellowship I volunteered as a full-time physics teacher to high school seniors} \\ %\\

\vspace{5pt} \\

Dec 2008 -     & Science Bowl Coach, Supervisor: Donald Austin, 518-388-6609 \\
 Feb 2009      & Kenney Community Center, 807 Union St., Schenectady, NY, 12305 \\ %\\

\vspace{5pt} \\

Nov 2007 -     & Planetarium Educator, Supervisor: Steve Russo, 518-382-7890 \\
 Jun 2009      & Suits-Bueche Planetarium, 15 Nott Terrace Heights, Schenectady, NY, 12308 \\ %\\

\vspace{5pt} \\

Jan 2007 -     & Physics Help Center Tutor, Supervisor: Jay Newman, 518-388-6506 \\
 Mar 2009      & Union College Physics Department, 807 Union St., Schenectady, NY, 12308 \\

\end{tabulary}

\endgroup


% %----------------------------------------------------------------------------------------
% %	Notable Presentations
% %----------------------------------------------------------------------------------------
% 
% \section{Notable Presentations}
% 
% \begin{tabulary}{0.9\textwidth}{RL}	
% 
% 2018 & CMS Exotica Workshop, Athens, Greece,
%  \emph{Search for Higgs boson decays to long-lived scalar particles in associated Higgs boson production} \\
% 
% 2018 & February CMS Week Run Coordination session, CERN, Meyrin, Switzerland,
%  \emph{HCAL endcap upgrade status} \\
% 
% 2017 & December CMS Week Run Coordination session, CERN, Meyrin, Switzerland,
%  \emph{HCAL end of year summary} \\
% 
% 2017 & June CMS Week Run Coordination session, CERN, Meyrin, Switzerland,
%  \emph{Status of the HCAL} \\
% 
% 2017 & June CMS Week HCAL session, CERN, Meyrin, Switzerland,
%  \emph{The Hadronic Calorimeter in the first half of 2017}, chair of session \\
% 
% 2016 & CMS Exotica Workshop, Zurich, Switzerland, 
%  \emph{Monophotons post ICHEP - a summary of 2016} \\ 
%  % https://indico.cern.ch/event/571620/other-view?view=cms
% 
% 2016 & CERN Internal Preapproval Talk for SMP-16-004, CERN, Meyrin, Switzerland, 
%  \emph{Measurement of the $Z(\nu\nu)\gamma$ production cross section at $\sqrt{s}=$13 TeV} \\
% 
% 2016 & Posters@LHCC (poster), CERN, Meyrin, Switzerland, 
%  \emph{Measurement of the W boson production cross section in association with two
%  b jets in pp collisions at $\sqrt{s}=$8 TeV} \\
% 
% 2015 & CERN Internal Approval Talk for SMP-14-020, CERN, Meyrin, Switzerland, 
%  \emph{Measurement of the W boson production cross section in association with two
%  b jets in pp collisions at $\sqrt{s}=$8 TeV} \\
% 
% 2015 & XXIII International Workshop on Deep-inelastic scattering and related subjects, Southern Methodist University, Dallas, TX, USA, 
%  \emph{Vector boson production in association with jets and heavy flavor quarks at CMS} \\
% 
% 2015 & Physics Department Colloquium, Union College, Schenectady, NY, USA, 
%  \emph{Experimental high-energy particle physics at CMS} \\
% 
% 2015 & American Physical Society April Meeting, Baltimore, MD, USA, 
%  \emph{Measurement of the $W+b\overline{b}$ cross section at CMS} \\
% 
% 2014 & Posters@LHCC (poster), CERN, Meyrin, Switzerland,
%  \emph{Study of charm and bottom production in association with a W boson at CMS } \\
% 
% 2011 & American Astronomical Society (poster), University of Washington, Seattle, WA, USA,
%  \emph{Multi-frequency optical-depth maps and the case for free-free absorption 
%  in two candidate compact symmetric objects: 1321+410 and 0026+346} \\
% 
% 2010 & Astronomical Society of New York, Rensselaer Polytechnic Institute, Troy, NY, USA,
%  \emph{VLBI imaging of two compact symmetric objects } \\
% \end{tabulary}
% 
% 
% 
% 
% 
% % \section{Academic/Employment History}
% % 
% % I am currently teaching algebra at Santa Fe Preparatory School as well as learning about quantitative genetics through the University of Wisconsin and auditing the Andrew Ng machine learning course at Stanford. For both of these courses, I am doing the associated homework and uploading code to \href{https://github.com/tmrhombus/CodingPractice}{github}. 
% % 
% % I am currently in the second half of writing a book relating to particle physics and the discovery of the Higgs boson with over 400 pages completed in draft form. Previously, I was at the Santa Fe Preparatory School, teaching and developing a curriculum for Calculus and Algebra II. Prior to that, I was a carpenter at South Shore Boatworks under the supervision of my uncle Bob Fuller, a third generation master craftsman and owner of the last shop in the USA still producing all-wood traditional ship wheels. In addition to wheels, I worked on custom marine joinery from the finishing of interiors for larger ships to complete builds of smaller craft under power or sail as well as developing a system for incorporating modern CNC routers into the traditional hand system.
% % 
% % \vspace{2pt}
% % 
% % Before coming to South Shore Boatworks, I was based at the European Organization for Nuclear Research (CERN) for nearly six years, where I earned my PhD in particle physics and worked as a postdoctoral research fellow for two years. During the first year of my postdoc I was selected to skip the first tier of management and became directly the Operations Manager for the Hadronic Calorimeter (HCAL), one of four subsystems comprising the Compact Muon Solenoid (CMS) experiment. In this role, I was responsible for overseeing the testing, installation, and commissioning of new readout electronics on both endcaps of CMS, chairing weekly meetings discussing the status and plans for the HCAL, leading training sessions on operations/data taking/technical debugging, reporting to upper management on behalf of the HCAL, and carrying a CERN phone at all times so as to be able to respond to interruptions in data taking at any time. 
% % 
% % \vspace{2pt}
% % 
% % As a postdoc I also supervised graduate students on a new physics analysis searching for long-lived scalar particles arising as the product of Higgs boson decays using the associated production of a vector boson to identify candidate events. We used central CMS software based in \texttt{c++/ROOT/python} for initial object reconstruction and I wrote a framework from scratch using these languages to do the selections/analysis/plotting which was later more generally adopted by others and is still in use today.
% % 
% % \vspace{2pt}
% % 
% % I came to the University of Wisconsin - Madison as a graduate student in the summer of 2011 after graduating in 2009 from Union College and spending a year in rural Uganda on a social entrepreneurship fellowship and another working as a technician for a chemical engineer at the USA site of General Electric's Global Research Center. Before starting my graduate school course work, I was with IceCube for a summer and began research with the CMS collaboration under Wesley Smith in the beginning of 2012, coming out to CERN for the first time that summer and taking shifts monitoring the data taking for the end of Run 1. After finishing my second year back in Madison, I moved to CERN full-time in May 2013 and contributed studies which were used in the Technical Design Report for a new trigger system as well as working on the 7 TeV W+bb physics analysis.
% % 
% % During my time as a graduate student, I continued on to become the editor for the 8 TeV W+bb analysis and was selected as the point of contact between the Standard Model Physics analysis groups and the b-tagging group. As the LHC was preparing to turn back on in 2015, I regularly took shifts in the CMS control room, and often worked underground and in the lab testing/preparing/installing hardware for the upgraded trigger system. As the 8 TeV W+bb paper began nearing completion, I joined and the monophoton analysis and contributed studies for a year before the paper went public. Unfortunately, we did not find dark matter. 

\end{document}
